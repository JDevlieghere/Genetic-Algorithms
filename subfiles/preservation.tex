\documentclass[../main.tex]{subfiles}

\begin{document}


This section covers selection and replacement techniques in the context of a crossover-based evolutionary algorithms. Their purpose is to introduce selection after reproduction in a way that checks whether or not crossover and mutation were able to produce a new solution candidate that outperforms its parents.

\begin{itemize}
    \item \textbf{OS:} A certain ratio of the next generation has to consist of children that can outperform their parents.
    \item \textbf{RAPGA:} new child solutions are added to the population as long as it is possible to generate unique and successful offspring from the gene pool of the last generation.
\end{itemize}

An upper limit for the selection pressure provides a good termination criterion.

\section{Offspring Selection (OS)}
\sectionmark{OS}

The progression of genetic search is the assured by means of successful offspring. The goal of OS is to create a sufficient number of children that surpass their parents' fitness.

\subsection{Success Ratio}
The success ration is defined as the relation between the next population members that need to be generated by successful mating, in relation to the total population.

\begin{equation}
    \text{SuccRation} \in [0,1]
\end{equation}

Having filled up the the claimed ratio, the rest of the generation is filled with randomly generated individuals that were also created by crossover but did not reach the criteria.

\subsection{Comparison Factor}
In oder to classify a child as \emph{better} than it's parent, the comparison factor is introduced. A comparison factor of 0 means that the child is considered better when it outperforms its weakest parent while it has to outperform its best parent for a comparison factor of 1.

\begin{equation}
    \text{CompFactor} \in [0,1]
\end{equation}

The comparison factor is scaled from 0 to 1 during the run of the algorithm. This causes a broader search at the beginning and a more directed search in the end.

\subsection{Actual Selection Pressure}
The actual selection pressure is the quotient of individuals that had to be considered until the success ration was reached, and the number of individuals in the population in the following way:

\begin{equation}
    \text{ActSelPress} = \frac{ | \text{POP}_{i+1} | + | \text{POOL} | }{ | \text{POP}_i | }
\end{equation}

An upper limit defines the maximum numbers of offspring considered. When this limit is sufficiently high, the model can be used for detecting premature convergence.

\textit{If it is no longer possible to find a sufficient number $(\text{SuccRatio} \dot | \text{POP} |)$ of offspring, outperforming their own parents even if $(\text{MaxSelPress} \dot | \text{POP} |)$ candidates have been generated, premature convergence has occurred.}

Higher success ratios cause higher selection pressure. This does not necessary cause premature convergence because the new selection step does not accepts clones. The latter are a major reason for premature convergence in regular GAs.

\section{Relevant Alleles Preserving GA (RAPGA)}
\sectionmark{RAPGA}

The goal of this enhanced algorithm variant is trying to bring out as much progress from the actual generation as possible and losing as little genetic diversity as possible at the same time. This is achieved by adjusting the population size. Potential offspring are accepted as members of the next generation $\Leftrightarrow $ they are able to outperform the fitness of their parents and if they are new in the sense that their chromosome consists of a concrete allele alignment that is not represented yet in the  an individual of the current generation.

\subsection{Considerations}

The following practical aspects of the RAPGA need to be considered:

\begin{itemize}
    \item The algorithm should offer different parent selection mechanisms, even for different parents (male and female) or allow for it to be disabled completely (random).
    \item It is reasonable to have more than one crossover and mutation operator. This works even for operators only generating a good result sporadically, because only successful chromosomes are considered in the evolutionary progress.
    \item The population size requires a lower and upper bound. The upper limit is required to prevent snowballing in the first rounds. The lower limit is needed to have a sufficient amount of chromosomes present to outperform parents. Additionally, it can act as a detector for convergence.
    \item Checking genotipical identity prevents structurally identical individuals to be included in the next generation.
    \item The maximum effort per generation is the maximum number of newly generated chromosomes per generation, no matter whether they are accepted or not. This can be used to terminate generation rounds.
\end{itemize}

\end{document}