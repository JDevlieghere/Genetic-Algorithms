\documentclass[../main.tex]{subfiles}

\begin{document}

\begin{mdframed}
\fullcite{Janikow}
\end{mdframed}

\begin{abstract}
Supervised learning in attribute-based spaces is one of the most popular machine learning problems studied and,
consequently, has attracted considerable attention of the genetic algorithm community. The fullmemory approach
developed here uses the same high-level descriptive language that is used in rule-based systems. This allows for an
easy utilization of inference rules of the well-known inductive learning methodology, which replace the traditional
domain-independent operators and make the search task-specific. Moreover, a closer relationship between the underlying
task and the processing mechanisms provides a setting for an application of more powerful task-specific heuristics.
Initial results obtained with a prototype implementation for the simplest case of single concepts indicate that genetic
algorithms can be effectively used to process high-level concepts and incorporate task-specific knowledge. The method
of abstracting the genetic algorithm to the problem level, described here for the supervised inductive learning, can be
also extended to other domains and tasks, since it provides a framework for combining recently popular genetic
algorithm methods with traditional problem-solving methodologies. Moreover, in this particular case, it provides a very
powerful tool enabling study of the widely accepted but not so well understood inductive learning methodology.
\end{abstract}

\end{document}