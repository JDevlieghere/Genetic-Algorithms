\documentclass[../main.tex]{subfiles}

\begin{document}

\begin{abstract}
The task of optimizing a complex system presents at least two levels of problems for the system designer. First, a
class of optimization algorithms must be chosen that is suitable for application to the system. Second, various
parameters of the optimization algorithm need to be tuned for efficiency. A class of adaptive search procedures called
genetic algorithms (GA) has been used to optimize a wide variety of complex systems. GA's are applied to the second
level task of identifying efficient GA's for a set of numerical optimization problems. The results are validated on an
image registration problem. GA's are shown to be effective for both levels of the systems optimization problem.
\end{abstract}

\begin{mdframed}
\fullcite{4075583}
\end{mdframed}

\section{Introduction}

When considering optimization problems, different techniques can be considered. Often, the choice of parameters that
must be tuned have a significant impact on the effectiveness of the method. This problem represents a secondary
optimization, on the meta-level. Here, the GA class of optimization algorithms is considered for that purpose.

The paper describes experiments that search a parametrized space of GA's in order to identify efficient GA's for the
task of optimizing a set of numerical functions. This search is performed by a meta-level GA. It is shown that genetic
algorithms are suitable for both levels of optimization.

\section{Experimental Design}
The searches for the optimal GA's were performed by GA's. This demonstrates the efficiency and power of GA's as meta-
level optimization techniques. A meta-level GA could similarly search any other space of parameterized optimization
procedures.

\section{Space of Genetic Algorithms}

This study is limited to a particular subclass of GA's characterized by the following six parameters.

\begin{enumerate}
	\item \textbf{Population Size (N)} \\
	A large population is more likely to contain representatives from a larger number of hyperplanes. Thus, it performs
	a more informed search. However, a large population requires more evaluations, possibly causing an unacceptably slow
	rate of convergence.

	\item \textbf{Crossover Rate (C)} \\
	The crossover rate controls the frequency of applying this operator. A higher rate leads to faster introduction of
	new structures but can cause them to be discarded before selection can produce improvements. A too low rate may
	cause stagnation.

	\item \textbf{Mutation Rate (M)} \\
	Mutation increases the variability of the population. Approximately $M \cdot N \cdot L$ mutations occur per
	generation. It intents to prevent forever converting to the same local optimum.

	\item \textbf{Generation Gap (G)} \\
	The generation gap determines what percentage of the population is replaced each generation. Thus, $N \cdot (1-G)$
	individuals survive intact in the next generation.

	\item \textbf{Scaling Window (W)} \\
	The performance value of a structure $x$ is defined as $u(x) = f(x) - f_{\text{min}}$. Here $f_{\text{min}}$ is the
	minimum value $f$ can achieve in the search space. This transformation guarantees that that the performance is
	positive. Because this is usually not know a priori, it can be substituted by $f(x_{\text{min}})$. Unfortunately,
	it makes distinguishing good values of $x$ hard. Three scaling modes are considered, based on the parameter $W$.

	\begin{itemize}
		\item If $W = 0$, then $f'_{\text{min}}$ is set to the minimum $f(x)$ in the first generation. In succeeding
		generations, all structure that evaluate to less than $f'_{\text{min}}$  are ignored. The value of
		$f'_{\text{min}}$ is updated when \textbf{all} structures in the population have evaluations greater than
		$f'_{\text{min}}$.
		\item If $0 < W < 7$, then we set $f'_{\text{min}}$ to the least value of $f(x)$ in the last $W$ generations.
		\item If $W = 7$, an infinite window is specified and no scaling is performed.
	\end{itemize}

	\item \textbf{Selection Strategy (S)} \\
	Two selection strategies are considered.
	\begin{itemize}
		\item \textbf{Pure Selection Strategy} \\
		If $S = P$, then a pure selection strategy is used, where each structure in the current population is
		reproduced a number of times proportional to its performance.
		\item \textbf{Elitist Selection Strategy} \\
		If $S = E$, then an elitist selection strategy is used. First, pure selection is performed. The best performing
		structures survive intact into the next generation.
	\end{itemize}
\end{enumerate}
A genetic algorithm can be identified by these parameters.
\[
\text{GA} = (N,C,M,G,W,S)
\]
The Cartesian product of the indicated ranges for the six parameters define a space of $2^{18}$ GA's. Sometimes it is
possible to predict how a single parameter will after the performance, assuming all others are kept fixed. Generally,
it is difficult to predict their interaction.

\subsection{Performance Measures}

Two performance measures for adaptive search strategies were considered: \textbf{online performance} and
\textbf{offline performance}. The former is a relevant measure when the search can be performed offline, while the best
structure so far is used to control the latter.

\section{Results}
In summary, the performance of GA's appear to be a nonlinear function of the control parameters. Nevertheless, the
available data is too limited to confirm or disconfirm the existence of discontinuities or multiple local optima in the
performance space of GA's.

\begin{itemize}
	\item Mutation rates above $0.05$ are generally harmful with respect to online performance. Performance approaches
	random search with a value grater than $0.10$, regardless of other parameter settings.
	\item Best online performance can be obtained with a population size in the range $30 - 100$.
	\item A large generation gap generally improves performance, as does the elitist strategy.
	\item The adoption of a small scaling window (1-5 generations) is associated with a slight improvement in both
	online and offline performance. Scaling enhances the pressure in the later stages of the search.
	\item For small populations (20 to 40) structures, good online performance is associated with either high crossover
	rate + low mutation rate or low crossover rate + high mutation rate.
	\item For mid-sized populations (30 to 90 structures), the optimal crossover rate appears to decrease as the
	population size increases.
\end{itemize}

\section{Conclusion}
The experimental data also suggests that, while it is possible to optimize GA control parameters, very good
performance can be obtained with a range of GA control parameter settings. An alternative approach to the optimization
of GA's would be to enable the GA to modify its own parameters dynamically
during the search.

\section{TLDR;}
GA can be used to search for optimal parameters for optimization problems. In particular a metalevel GA can be used to select optimal GAs (with optimal parameter settings) from a space of available GAs. 

Using standard operators and representation, few GAs perform better than SGA. Parameter settings that were shown to be beneficial : 
\begin{itemize}
\item Mutation rate should generally be higher than 0\% but lower than 5\%.
\item Optimal crossover rate depends on the population size, lower for large population.
\item A small scaling window almost always helps.
\item Elitism should be present.
\item A large generation gap is shown to be optimal.
\end{itemize}

\end{document}