\documentclass[../main.tex]{subfiles}

\begin{document}

\begin{mdframed}
\fullcite{Michalewicz:1996:GAD:229930}
\end{mdframed}

There are 3 constraint-handling techniques considered in this chapter.

\begin{enumerate}
	\item \textbf{Penalty:} Generate solutions without considering constraints but decrease their ``goodness'' in the
	evaluation function. The problem is converted to an unconstrained problem with a penalty function. The penalty can
	either be constant or in function of the violation. The extreme is eliminating non-feasible solution with a death-
	penalty.
	\item \textbf{Repair:} Correct the infeasible solutions. However, they might be computationally intensive and
	require to be tailored to the application.
	\item \textbf{Decoder:} Use a special representation mapping (decoder) which guarantees the generation of a
	feasible solution. Alternatively use problem-specific operations. Frequently, this might not be feasible for all
	constraints and is rather computationally intensive.
\end{enumerate}

\section{The 0/1 Knapsack Problem}

The knapsack problem is used to evaluate the 3 techniques. The problem consists of a capacity limited bag, and items
with weights and profits. The goal is to maximize the profit under the constraint that the sum of weights cannot exceed
the capacity of the sack.

\section{The Test Data}

The difficulty of the problem is greatly affected by the correlation between profits and weights. We consider 3 cases.

\begin{enumerate}
	\item \textbf{Uncorrelated:} Both the weight and profit are distributed uniformly random ($[1 \mathellipsis v]$).
	\item \textbf{Weakly correlated:} The weight is distributed uniformly random. Profits are associated with weights
	by a random factor ($[1 \mathellipsis r]$).
	\item \textbf{Strongly correlated:} The weight is distributed uniformly random. Profits are associated with weight
	by a fixed factor ($r$).
\end{enumerate}

Higher correlations implies a smaller value of the difference:

\begin{equation}
\text{max}_{i=1 \mathellipsis n} { P[i]/W[i]} - \text{min}_{i=1 \mathellipsis n} { P[i]/W[i]}
\end{equation}

Higher correlation problems have higher expected difficulty. Two knapsacks are considered:

\begin{itemize}
	\item \textbf{Restrictive Knapsack Capacity:} Here $C_1 = 2v$. The optimal solution contains very few items.
	\item \textbf{Average Knapsack Capacity:} Here $C_2 = \frac{1}{2} \sum^n_{i=1}W[i]$. About half the items are in
	the optimal solution.
\end{itemize}

\section{Algorithms}

\subsection{Penalty Algorithms}

The solution is represented by a binary string of length $n$: the $i$-th item is selected for the knapsack when $x[i] =
1$. The fitness is given by
\begin{equation}
\text{eval}(x) = \sum^n_{i=1} x[i] \cdot P[i] - \text{Pen}(x)
\end{equation}
Three different strategies for assigning penalties are considered. In all cases $\rho = \text{max}_{i=1 \mathellipsis
n} { P[i]/W[i]}$.

\begin{itemize}
	\item \textbf{Logarithmic} \\
	\[
	A_p[1]: \text{Pen}(x) = log_2(1 + \rho \cdot (\sum^n_{i=1} x[i] \cdot P[i] - C))
	\]
	\item \textbf{Linear}
	\[
	A_p[2]: \text{Pen}(x) = \rho \cdot (\sum^n_{i=1} x[i] \cdot P[i] - C)
	\]
	\item \textbf{Quadratic}
	\[
	A_p[1]: \text{Pen}(x) = (\rho \cdot (\sum^n_{i=1} x[i] \cdot P[i] - C))^2
	\]
\end{itemize}


\subsection{Repair Algorithms}

The same representation is used as in the previous section. The evaluation function is very similar but lack the
penalty function.

\begin{equation}
\text{eval}(x) = \sum^n_{i=1} x[i] \cdot P[i]
\end{equation}

Two repair algorithms are considered that only differ in the selection procedure.

\begin{itemize}
	\item \textbf{Random Repair} $A_r[1]$: A random item is removed from the knapsack until it is no longer overfilled.
	\item \textbf{Greedy Repair} $A_r[2]$: All items in the knapsack are sorted in decreasing order of profit-to-weight
	ratio. The last items are removed from the knapsack until it is no longer overfilled.
\end{itemize}

\subsection{Decoder Algorithms}

The ordinal representation is used for representing chromosomes. The $i$-th element is always in the range of $[1
\mathellipsis n - i + 1]$. The one-point crossover operator applied to any valid chromosome will generate a valid
solution. Two build procedures are considered:

\begin{itemize}
	\item \textbf{Random Decoding} $A_d[1]$: The representation is created such that the order of items corresponds
	with the order in the input.
	\item \textbf{Greedy Decoding} $A_d[2]$: The representation is created such that the order of items is decreasing
	in the profit-to-weight ratio.
\end{itemize}

\section{Results}

\begin{itemize}
	\item Penalty functions $A_p[i]$ do not produce feasible results for problems with restrictive capacity ($C_1$).
	\item Considering average capacity ($C_2$) problems, the logarithmic penalty function outperforms all others.
	\item Considering restrictive capacity ($C_2$) problems, the greedy repair method $A_r[2]$ outperforms all other
	methods.
\end{itemize}

The results are intuitive: for the restrictive problem, only a small fraction constitute the feasible solution. The
smaller the ratio between the feasible part and the search space, the harder for the penalty methods to provide
feasible results.

\end{document}