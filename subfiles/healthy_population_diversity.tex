\documentclass[../main.tex]{subfiles}

\begin{document}

\begin{mdframed}
\fullcite{6036171}
\end{mdframed}

\begin{abstract}
This paper presents ACROMUSE, a novel genetic algorithm (GA) which adapts crossover, mutation, and selection
parameters. ACROMUSEs objective is to create and maintain a diverse population of highly-fit (healthy) individuals,
capable of adapting quickly to fitness landscape change and well-suited to the efficient optimization of multimodal
fitness landscapes. A new methodology is introduced for determining standard population diversity (SPD) and an original
measure of healthy population diversity (HPD) is proposed. The SPD measure is employed to adapt crossover and mutation,
while selection pressure is controlled by adapting tournament size according to HPD. In addition to selection pressure
control, ACROMUSE tournament selection selects individuals according to healthy diversity contribution rather than
fitness. This proposed selection mechanism simultaneously promotes diversity and fitness within the population. The
performance of ACROMUSE is evaluated using various multimodal benchmark functions. Statistically significant results
are presented comparing ACROMUSEs fitness and diversity performance to that of several other GAs. By maintaining a
diverse population of healthy individuals, ACROMUSE responds to fitness landscape change by restoring better fitness
scores faster than other GAs. Analysis of the adaptive operators illustrates that the key benefit of ACROMUSE is the
synergy of the operators working together to achieve an effective balance between exploration and exploitation.
\end{abstract}

Premature population convergence about a local optimum is a common problem. It is usually caused by one of the following reasons.

\begin{enumerate}
	\item \textbf{Incorrect application of selection pressure:} Whereby a ``super-performer'' in the population
	dominates the selection procedure and eliminates much of the population's diversity.
	\item \textbf{Too low mutation rate:} diversity is lost trough selection and not recovered via mutation.
	\item \textbf{Loss of crossover efficacy:} As a population becomes more converged, recombination of similar individuals causes similar offspring (inbreeding).
\end{enumerate}

This papers presents \textbf{ACROMUSE} which adapts the crossover, mutation and selection parameters. The main goal is maintaining a diverse population of healthy individuals. It employs two measures to of population diversity.

\begin{itemize}
	\item \textbf{Standard Population Diversity (SPD)} \\
	This measure describes the population's solution space diversity. The SPD measure is used to control mutation and crossover. Crossover employs SPD to divide the population into an \textbf{exploration section} and an \textbf{exploitation section}. The sizes of these sections are controlled by the SPD measure.
	\item \textbf{Healty Population Diversity (HPD)} \\
	This measure describes a population’s solution space diversity from a fitness perspective, i.e., a measure of the
	diversity of healthy individuals. HPD is used to regulate selection pressure. Tournament size is reduced when HPD is low permitting lower-fitness individuals to reproduce.
\end{itemize}

Maintaining a diverse population is very important for GA search. Not only does high-genetic diversity increase the
population's search coverage but it also endows the population with a degree of robustness in case of environmental change.

\subsection{SPD \& HPD}

\subsubsection{Standard Population Diversity (SPD)}
The SPD describes the level of variation in a population. Genetic diversity is a very important component of
evolutionary exploration since a GA can only search the space offered to it by the genes present in the population.

\subsubsection{Calculating SPD}
\subsubsection{Healthy Population Diversity (HPD)}
\subsubsection{Calculating HPD}


\end{document}