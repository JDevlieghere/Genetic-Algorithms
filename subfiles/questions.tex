\documentclass[../main.tex]{subfiles}

\begin{document}

\begin{enumerate}
    \question{Explain Holland's schema theorem. Which assumptions that are made in this theorem and are somewhat
    unlikely?}{ The schema theorem considers the canonical GA with binary string representation of individuals with an
    alphabet $\{0,1,\#\}$ where the latter is a wild card symbol. The theorem says that the number of short schemata
    with low order and above average quality, grows exponentially in subsequent generations of a genetic algorithm. This
    result is obtained by considering the influence of replication, crossover and mutation. It is unlikely because it
    is obtained under idealized conditions that do not hold for most practical GA applications. Furthermore, very few
    problems can be modeled as the canonical GA (binary encoding, proportional selection, single-point crossover,
    strong mutation).
    }
    \question{Explain, chapter 4, page 69, the sentence ``Both algorithms are self-adaptive \textellipsis'' (About Offspring Selection and RAPGA)}{\unanswered}
    \question{Suppose you want to solve an optimization problem, for which the objective function (function to be maximized) is known. Why it is usually not appropriate to use the objective function as the fitness function? Give several reasons if possible. How can you construct a good fitness function? Give an example of such a construction of a fitness function in one of the texts that we have discussed.}{\unanswered}
    \question{Given the charts in the article on GABIL, can you say that the tree based concept learner (ID5R) outperforms the genetic algorithm? Discuss figures 1 -- 7 in detail.}{\unanswered}
    \question{Explain the terms ``epistasis'' and ``deception''. What is the difference?}{\unanswered}
    \question{Discussion of rank-based selection. Which parameters can be used to adapt the behavior of these methods?
    When would you want to use a rank-based selection method?}{ In the context of linear-rank selection the individuals
    of the population are ordered according to their fitness and copies are assigned in such a way that the best
    individual receives a pre-determined multiple of the number of copies the worst one receives. On the one
    hand rank selection implicitly reduces the dominating effects of ``super individuals'' in populations (i.e.,
    individuals that are assigned a significantly better fitness value than all other individuals), but on the other
    hand it warps the difference between close fitness values, thus increasing the selection pressure in stagnant
    populations. Even if linear-rank selection has been used with some success, it ignores the information about
    fitness differences of different individuals and violates the schema theorem}
    \question{Discussion of the paper Evolving 3D morphology and behavior by competition by K. Sims. The fitness function is different from the fitness functions we are used to. Explain how and why.}{\unanswered}
    \question{Handling constraints: Explain the difference between algorithms based on repair methods and algorithms based on decoders. Why these strategies can lead to a better performance compared with algorithms based on penalty functions? How would you take into account the constraints arising in a timetabling problem?}{\unanswered}
    \question{What's the importance of the schemata theorem? Given a schema where the fitness is 20\% above the average. Can you discuss how long it will take until the whole population contains this schema?}{\unanswered}
    \question{What techniques are used in order to obtain better load balance? Can these be used in other applications?}{\unanswered}
    \question{Discuss some drawbacks of the roulette wheel selection mechanism, and describe some methods to avoid (or to minimize) these drawbacks.}{
    Fitness proportionate select can cause premature convergence because outstanding individuals quickly take over the entire population. The dominance of a single group of highly fit individuals (``super individuals'') can be reduced by stochastic sampling techniques. The second drawback is caused when the fitness values are close together,leading to low selection pressure. This can be mitigated by so called ``windowing techniques'' to make the selection independent of the dimension of the fitness value.
    }
    \question{What is evolutionary programming? Can this still be considered a genetic algorithm? Give one ore two examples to illustrate the approach.}{\unanswered}
    \question{Consider varying population sizes for GA's. Discuss (1) the purpose, (2) the importance of the life parameter and (3) the relevant figure in the book.}{\unanswered}
\end{enumerate}

\end{document}