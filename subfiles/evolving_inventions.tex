\documentclass[../main.tex]{subfiles}

\begin{document}

\begin{mdframed}
\fullcite{Koza2003}
\end{mdframed}

\begin{abstract}

Evolution is an immensely powerful creative process. From the intricate biochemistry of individual cells to the
elaborate structure of the human brain, it has produced wonders of unimaginable complexity. Evolution achieves these
feats with a few simple processes--mutation, sexual recombination and natural selection--which it iterates for many
generations. Now computer programmers are harnessing software versions of these same processes to achieve machine
intelligence. Called genetic programming, this technique has designed computer programs and electronic circuits that
perform specified functions.

In the field of electronics, genetic programming has duplicated 15 previously patented inventions, including several
that were hailed as seminal in their respective fields when they were first announced. Six of these 15 existing
inventions were patented after January 2000 by major research institutions, which indicates that they represent current
frontiers of research in domains of scientific and practical importance. Some of the automatically produced inventions
infringe squarely on the exact claims of the previously patented inventions. Others represent new inventions by
duplicating the functionality of the earlier device in a novel way. One of these inventions is a clear improvement over
its predecessor.

Genetic programming has also classified protein sequences and produced human competitive results in a variety of areas,
such as the design of antennas, mathematical algorithms and general-purpose controllers. We have recently filed for a
patent for a genetically evolved general-purpose controller that is superior to mathematically derived controllers
commonly used in industry.

\end{abstract}


In the field of electronics, GP duplicated existing, patented inventions. Resulting inventions were in some cases very
similar and in others completely square to current claims in the patent. First practical commercial area for GP is
design. Usually there are many trade-offs and genetic programming does not require insight in underlying structure.

\subsection{Methodology}

Starting from random generated organism. High level description of their function. Fitness is evaluated by mathematical
formula. Operations progressively improve the population.

\begin{itemize}
	\item \textbf{Crossover} (90\%): search in space of all possible organisms.
	\item \textbf{Mutation} (1\%): local search for advantage near existing good individuals.
	\item \textbf{Copying} (9\%): the best individuals, copied unaltered so new generation is at least as fit as
	previous.
\end{itemize}

\subsection{Applications}

\begin{ex}
A low-pass filter is used in a hi-fi system to send only the lowest frequencies to the woofer speaker. Circuit starts
as an embryo, consisting of a single wire running from the input to the output. The crossover and mutation operations
acting on numerical expressions will adjust component values so that the cutoff frequency approaches the desired 1,000
Hz.
\end{ex}

\begin{ex}
Small improvements in the ``tuning rules'' used in customizing a controller can result in large benefits. An example is
the proportional-integral-derivative controller (PID controller). This is a control loop feedback mechanism
(controller) widely used in industrial control systems and calculates  the error value as the difference between a
measured process variable and a desired set-point.
\end{ex}

\subsection{Hardware}
The evolutionary process requires efficiently evaluating the fitness of thousands or millions of offspring in each
generation. This can be achieved either by simulation or using rapidly reconfigurable field-programmable gate arrays.

\subsection{Run Time}
Evolution in nature thrives when organisms are distributed in semi-isolated subpopulations. The same seems to be true
of genetic programming run on a loosely connected network of computers. This can be achieved by migrating a small
percentage of individuals (selected based on fitness) ,at the end of each generation, to adjacent computers.

\subsection{Intelligence}
Genetic programming is delivering human-competitive machine intelligence with a minimum of human involvement for each
new problem and without using either logical deduction or a database of human knowledge. By being able to duplicate
existing, granted patents, it has passed the patent office's intelligence test.

\end{document}