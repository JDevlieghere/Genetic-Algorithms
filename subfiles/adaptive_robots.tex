\documentclass[../main.tex]{subfiles}

\begin{document}

\begin{mdframed}
\fullcite{Floreano:Evolution:2010}
\end{mdframed}

The paper illustrates how the process of natural selection can lead to the evolution of complex traits. After only a
few 100 generations, robots evolve to present collision-free movement, homing, sophisticated predator versus prey
behavior, co-adaptation of brain and bodies, cooperation and even altruism.

Robots are controlled by a simple neural network. Genes do not specify behavior directly but encode molecular products
that lead up to the development of brains and bodies.

\section{Principle of Selection}

The general idea of evolutionary robotics is to create a population with different genomes. Input neurons are activated
by the robot's sensors and the output neurons control the motors of the robot. Different neural networks lead up to
individual responses to sensory-motor interactions with the environment. Genomes are paired and mutation leads to new
genomes. The process is repeated over many generations until a stable behavioral strategy is established.

\section{Experiments}

\begin{experiment}
Darwinian selection has been used to generate small-wheeled robots that perform collision-free navigation. This
requires appropriate processing of sensor information and activation of the motors. The robot use 8 distance sensors
connected to 8 input neurons. After about $100$ generations, most of the robots exhibited collision-free navigation.
\end{experiment}

\begin{experiment}
This experiment was conducted to answer the question whether the robots could evolve the ability to find their way
home. The experiment was set up around a dark room with a light source at the home location. The robot's battery
discharged over 50 motor-cycles and was instantaneously regenerated once home. After 200 simulations, the best robot
explored the arena while only returning home when its batteries had approximately 10\% residual energy.
\end{experiment}

\begin{experiment}
Experimental evolution with robots has also been used to study the co-evolutionary processes between a population of
predator robots and a population of prey robots. Both are equipped with distance sensors. Prey robots are twice as
fast but the predator has an upgraded vision system that can perceive the black stick present on the prey. Preys
developed fast motion while predators would visually track them. Interestingly, because of the weak selection pressure
for wall avoidance, the predators would lose this ability. They had become so efficient that the prey would be caught
before hitting a wall.
\end{experiment}

\section{Evolution of Brains and Body Morphologies}
Experimental evolution has also been used to co-evolve artificial brains and morphologies of simulated robots. The work
of Sims \cite{Sims:1994:EMB:1667943.1667946} is discussed here. Please refer to this summary.

\section{Evolution of Cooperation and Altruism}
Experimental evolution was also used to investigate whether robots could develop cooperative and altruistic behavior.

\begin{experiment}
The experimental setup consisted of a foraging situation in a square arena containing ten sugar cube-sized wheeled
robots, small tokens that a single robot could push, and large tokens requiring at least two robots to be pushed. A
large token successfully pushed along the white wall increased the fitness of all robots within a group while a small
token successfully pushed increased the fitness of only the robot that pushed it.

In groups of unrelated robots, they invariably specialized in pushing the small objects. By contrast, the presence of
related robots within groups allowed the evolution of altruism.
\end{experiment}

\section{Conclusion}
These examples of experimental evolution with robots verify the power of evolution by mutation, recombination, and
natural selection. A major issue in evolutionary robotics is that agents may use idiosyncratic features of the
environment in which they are selected to increase performance, hence leading to a major fitness drop in new
environments where these features are lacking. Nevertheless, it provides a powerful means to study how phenotypes can
be shaped by natural selection and address questions that are difficult to address with real organisms.
\end{document}